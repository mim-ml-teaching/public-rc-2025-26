\documentclass{article}
\usepackage{amsmath}
\begin{document}

\section*{RC 2023/24, Exam, theory}

Every problem is worth the same amount of points. Every problem
should be solved on a separate sheet of paper. Write your name on
every sheet of paper. You can use a calculator and a ruler. You
have 60 minutes to solve the problems.


\subsection*{Problem 1}

Consider the non-linear dynamical system given by the following
equations:

\begin{align}
    \dot{x} & = y - x^3 + 3x^2 \\
    \dot{y} & = -x - 2y^3 + y
\end{align}

where \(\dot{x}\) and \(\dot{y}\) represent the time derivatives of \(x\) and \(y\) respectively.

\textbf{Tasks:}

\begin{enumerate}
    \item \textbf{Find the Fixed Points:} Determine all the fixed points of this system.

    \item \textbf{Linearize the System:} For each fixed point found in part (1), linearize the system around that point by computing the Jacobian matrix at each fixed point.
          \[
              J = \begin{pmatrix}
                  \frac{\partial \dot{x}}{\partial x} & \frac{\partial \dot{x}}{\partial y} \\
                  \frac{\partial \dot{y}}{\partial x} & \frac{\partial \dot{y}}{\partial y}
              \end{pmatrix}
          \]
\end{enumerate}

\subsection*{Problem 2}

Consider a robot with three joints and three links,
resembling the one encountered in Lab 13.
The first and last joints are revolute, while the second
joint is prismatic. The table below provides some of the
possible configurations ($\theta_1, l_2, \theta_3$) of the
robot along with the corresponding positions of the robot's
end effector:

\begin{tabular}{r|r|r|r|r|r}
    $\theta_1$ & $l_2$ & $\theta_3$ & $x$ & $y$ & $z$ \\
    \hline
    0          & 0     & 0          & 0   & 1   & 0   \\
    0          & 0.5   & 0          & 0   & 1.5 & 0   \\
    0          & 0.5   & 90         & 0.5 & 1   & 0   \\
    90         & 0.5   & 90         & 0   & 1   & 0.5 \\
    180        & 0.5   & 90         & ?   & ?   & ?   \\
    180        & 0.5   & 180        & ?   & ?   & ?   \\
\end{tabular}

We use degrees for the angles and meters for the lengths.

\textbf{Tasks:}

\begin{enumerate}
    \item \textbf{Missing values:} Find the missing values in the table above (write it on paper).
    \item \textbf{Forward kinematics:} Compute the forward kinematics of the robot end effector.
\end{enumerate}

\subsection*{Problem 3}
Let's assume that we have a pinhole camera with the following parameters:

\begin{itemize}
    \item Focal length: f = 1
    \item Principal point: c = (0, 0)
    \item Image resolution: 640 x 480
\end{itemize}

The camera, in world frame, is located at the position p = (0, 0, 0) and is oriented in the direction of the z axis.

The camera sees a point at the following pixel coordinates: p = (320, 240). z coordinate of the point in the world frame is z = 5.

\textbf{Tasks:}

\begin{enumerate}
    \item \textbf{Compute the 3D coordinates:} Compute the 3D coordinates of the point in the world frame.
    \item \textbf{Move the camera:} What would be the pixel coordinates of the point if the camera was moved to the position p = (0, 0, 1) remaining oriented in the direction of the z axis?
\end{enumerate}

Assume that the camera is calibrated and that the distortion is negligible.

\end{document}


